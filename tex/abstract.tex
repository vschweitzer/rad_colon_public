\section*{Abstract}

\paragraph{Purpose} To determine the predictability of \acs{pcr} in colorectal
tumor patients using statistical properties of associated \acs{mri} imaging data.
Through the usage of a standardized radiomics library, beneficial radiomics 
features for a better-than-random classification performance are determined. 
The implementation of this setup is attempted through the sole use
of widely available or replicable tools.

\paragraph{Experimental Setup} The dataset featuring inconsistent tumor annotations
was processed by a common semi-automatic annotation algorithm. An initial count 
of \featureCount{} radiomics features was extracted using PyRadiomics. Each 
feature was then assigned a weighted \enquote{importance} score, depending on 
its contribution to a successful classification. Finally, the ideal importance 
threshold was determined.

Additionally, the possibility of favorable feature combinations was explored by
creating groupings of related features. Models based on those groupings were
evaluated by comparing their actual performance against their expected 
performance. Feature combinations performing unexpectedly well were marked as 
potentially beneficial in combination.

Model performance was evaluated using a combination of balanced accuracy and 
\acs{roc} curve metrics. Measures were taken over the average of 100 classifier
models for any set of parameters.

\paragraph{Results} A classifier model reaching a balanced
accuracy of \num{0.697142857142857} (\SI{95}{\percent} Confidence Interval 
[\num{0.68272882292878}, \num{0.711556891356934}]) and an AUC mean of 
\num{0.766326530612245} (\SI{95}{\percent} Confidence Interval:
[\num{0.7529118302973336}, \num{0.7797412309271564}]) has been constructed.

Favorable groupings for increased model accuracy have been found.

\paragraph{Conclusion} In concordance with similar studies, the prediction of 
\acs{pcr} in colorectal tumor patients based on radiomics features seems 
possible.

Studies utilizing a different classification model for comparable challenges 
consistently performed better than the \acs{rf}-based classifier used here. 
Along with a strong difference in the amount of selected features, this result
possibly calls the effectiveness of the feature selection algorithm introduced 
in this thesis into question.

Although feature groupings that perform unusually well together have been found
as expected, low correlation between expected and achieved grouping performance
points to flaws in either the evaluation of groupings or the earlier mentioned 
weighting of feature importances.